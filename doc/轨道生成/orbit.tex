%orbit.tex 陨石和补给箱轨道的计算方法的说明

%	-* mode: XeLaTeX		encoding:UTF-8 *-
%	Copyright 2020 张子辰 & 吕航 (GitHub: WCIofQMandRA & LesterLv)
%
%	This file is part of the game 保卫行星
%
%	This game is free software; you can redistribute it and/or modify it
%	under the terms of the GNU Lesser General Public License as published by
%	the Free Software Foundation; either version 3, or (at your option) any
%	later version.
%
%	This game is distributed in the hope that it will be useful, but
%	WITHOUT ANY WARRANTY; without even the implied warranty of MERCHANTABILITY
%	or FITNESS FOR A PARTICULAR PURPOSE.  See the GNU Lesser General Public
%	License for more details.
%
%	You should have received copies of the GNU Lesser General Public License
%	and the GNU Gerneral Public License along with 保卫行星 .
%	If not, see https://www.gnu.org/licenses/.
%
%	This document itself is released under the GNU Free Documentation License.
%	
%	Permission is granted to copy, distribute and/or modify this document under 
%	the terms of the GNU Free Documentation License, Version 1.3 or any later 
%	version published by the Free Software Foundation; with no Invariant Sections, 
%	no Front-Cover Texts, and no Back-Cover Texts. A copy of the license is included
%	in the file named FDL-1.3.txt.
\documentclass[UTF8,fontset=none,linespread=1.2]{ctexart}
\xeCJKsetup{%
underdot={symbol={\fontsize{1.7em}{0pt}\selectfont%
\textrm{\textup{\textmd{\symbol{46}}}}}\relax}}
\usepackage{xunicode-addon}
\xeCJKDeclareCharClass{Default}{"24EA, "2460->"2473, "3251->"32BF,"24B6->"24E9,"2160->"217F}
\newfontfamily\EnclosedNumbers{Source Han Serif SC}
\AtBeginUTFCommand[\textcircled]{\begingroup\EnclosedNumbers}
\AtEndUTFCommand[\textcircled]{\endgroup}
\makeatletter
\let\textcircled@old\textcircled
\protected\def\textcircled#1{%
\expandafter\textcircled@old\expandafter{\expanded{#1}}}
\makeatother
\setCJKmainfont[ItalicFont={KaiTi},BoldItalicFont={KaiTi},
BoldItalicFeatures={FakeBold=3}]{Source Han Serif SC}
\setCJKsansfont[AutoFakeSlant,BoldItalicFeatures={FakeSlant}]
{Source Han Sans SC Normal}
\setCJKmonofont[AutoFakeBold=3,AutoFakeSlant]{FangSong}
\setmainfont{cmun}[Extension=.otf,UprightFont=*rm,
ItalicFont=*ti,BoldFont=*bx,BoldItalicFont=*bi]
\setsansfont{cmun}[Extension=.otf,UprightFont=*ss,
ItalicFont=*si,BoldFont=*sx,BoldItalicFont=*so]
\setmonofont{cmun}[Extension=.otf,UprightFont=*btl,
ItalicFont=*bto,BoldFont=*tb,BoldItalicFont=*tx]
\usepackage[a4paper,vmargin=2.5cm,hmargin=3cm]{geometry}
\usepackage{amsmath,amsfonts,amsthm,amssymb,upgreek,extarrows}
\DeclareMathOperator{\arctanh}{arccot}
\newcommand{\upd}{\mathrm{d}}
\newcommand{\upe}{\mathrm{e}}

\pagestyle{empty}

\begin{document}
\section*{行星运动的求解}
\noindent 先证一个引理:

\textit{圆锥曲线在顶点处的曲率半径等于该顶点与(最近的)焦点的距离.}

\noindent 证明:

在极坐标系下,设圆锥曲线的方程为$r=\dfrac{r_0}{1+\varepsilon\cos\theta}$.

顶点坐标为$\left(\dfrac{r_0}{1+\varepsilon},0\right)$.

任意一点曲率$$\kappa=\lim_{\Delta s\to 0}\dfrac{\Delta\phi}{\Delta s}
=\dfrac{\upd\theta}{\sqrt{r^2+\left(\frac{\upd r}{\upd\theta}\right)^2}\upd\theta}
=\dfrac{1}{\sqrt{r^2+\left(\frac{\upd r}{\upd\theta}\right)^2}}$$

又$\dfrac{\upd r}{\upd\theta}=\dfrac{r_0\varepsilon\sin\theta}{(1+\varepsilon\cos\theta)^2}$.

$\therefore$在顶点处$\kappa=\dfrac{1}{\sqrt{\left(\frac{r_0}{1+\varepsilon}\right)^2+0}}=\dfrac{1+\varepsilon}{r_0}$.

$\therefore$曲率半径$\rho=\dfrac{r_0}{1+\varepsilon}$.\hfill$\square$

\vspace{4ex}
设恒星的质量为$M$,万有引力常数为$G$.

轨道的极坐标方程可表示为$r=\dfrac{r_0}{1+\varepsilon\cos(\theta-\theta_0)}$,其中$\varepsilon$为
轨道离心率.

令$\cos(\theta-\theta_0)=1$得,近日点与恒星的距离$r_m=\dfrac{r_0}{1+\varepsilon}$.

近日点速率$v_m$满足$\dfrac{v_m^2}{r_m}=\dfrac{GM}{r_m^2}$.

$\therefore v_m=\sqrt{\dfrac{GM}{r_m}}=\sqrt{\dfrac{(1+\varepsilon)GM}{r_0}}$

设$h=v_mr_m=\sqrt{\dfrac{GMr_0}{1+\varepsilon}}$,由角动量守恒,行星在轨道上运动时$h=vr$不变.

单位时间行星与恒星连线扫过的面积$$\upd S=\dfrac12h\upd t=\dfrac12\sqrt{\dfrac{GMr_0}{1+\varepsilon}}\upd t=\dfrac12r^2\upd\theta$$

若以行星位于近日点时为计时起点,则行星运动到$(r,\theta)$处的时间$$t=\sqrt{\dfrac{1+\varepsilon}{GMr_0}}\int_{\theta_0}^{\theta}r^2\upd \theta=r_0^2\sqrt{\dfrac{1+\varepsilon}{GMr_0}}\int_0^{\theta-\theta_0}\dfrac{\upd x}{(1+\varepsilon\cos x)^2}$$

经计算:$$\int_0^{\theta}\dfrac{\upd x}{(1+\varepsilon\cos x)^2}=\left\lbrace\begin{array}{l}
-\dfrac{\varepsilon\sin\theta}{(1-\varepsilon^2)(\varepsilon\cos\theta+1)}+\dfrac{2\arctan\left(\sqrt{\frac{1-\varepsilon}{1+\varepsilon}\tan\frac\theta2}\right)}{(1-\varepsilon^2)^{\frac32}},\ 0<\varepsilon<1\\
\dfrac12\tan\dfrac\theta2+\dfrac16\tan^3\dfrac\theta2,\ \varepsilon=1\\
\dfrac{\varepsilon\sin\theta}{(\varepsilon^2-1)(\varepsilon\cos\theta+1)}-\dfrac{2\arctanh\left(\sqrt{\frac{\varepsilon-1}{\varepsilon+1}\tan\frac\theta2}\right)}{(\varepsilon^2-1)^{\frac32}},\ \varepsilon>1\\
\end{array}\right.$$

于是已知$\theta$、$r$、$t$中的任意一个,可求另外两者.
\end{document}